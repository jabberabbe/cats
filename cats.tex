% !TEX TS-program = LuaLaTeX
% !TEX encoding = UTF-8 Unicode
\documentclass[a5paper,oneside,11pt]{article}
\usepackage[a5paper,margin=1.5cm]{geometry}
\usepackage[pdfusetitle]{hyperref}
%\usepackage{fontspec}
%\usepackage{unicode-math}
\usepackage{tikz-cd}
\usetikzlibrary{cd}
\usepackage{float}
\usepackage{graphicx}
\usepackage{amsmath}
\usepackage{amssymb}
\usepackage{amsthm}

\title{Category theory notes and exercises}
\author{Tito Sacchi}

\newtheorem{ex}{Exercise}
\newtheorem{defn}{Definition}
\newtheorem{thm}{Theorem}
\newcommand\id{\mathord{\mathrm{id}}}
\newcommand\Id{\mathord{\mathrm{Id}}}
\newcommand\1{\mathord{\mathrm{1}}}
\newcommand\cat[1]{\mathbf{#1}}
\newcommand\op{\mathrm{op}}
\newcommand\Hom[3]{\mathop{\mathsf{Hom}_{\cat{#1}}(#2, #3)}}
\newcommand\dom{\mathop{\mathsf{dom}}}
\newcommand\cod{\mathop{\mathsf{cod}}}
\begin{document}
\maketitle

\section{Categories and morphisms}
\begin{defn}
\label{defn:monoepi}
A morphism $f : X \rightarrow Y$ is:
\begin{itemize}
    \item a \textbf{monomorphism} if $ f \circ g_1 = f \circ g_2 \implies g_1 = g_2$
        for any $g_1, g_2 : W \rightarrow X$ (we write $f : A \rightarrowtail B$);
    \item a \textbf{epimorphism}  if $ g_1 \circ f = g_2 \circ f \implies g_1 = g_2$
        for any $g_1, g_2 : Y \rightarrow Z$ (we write $f : A \twoheadrightarrow B$);
    \item a \textbf{isomorphism} if there exists another morphism $f^{-1} : Y \rightarrow X$
        such that $f^{-1} \circ f = \id_X$ and $f \circ f^{-1} = \id_Y$ (we write $X \simeq Y$).
\end{itemize}
\end{defn}
\begin{defn}
\label{defn:split}
Two morphisms $s : X \rightarrow Y$ and $r : Y \rightarrow X$ such that
$r \circ s = \id_A$ express $B$ as a \textbf{retract} of $A$. $s$ is
a \textbf{split monomorphism} and a right inverse or section to $r$,
while r is a \textbf{split epimorphism} and a left inverse or retraction to $s$.
$s \circ r$ is a \textbf{split idempotent}.
\end{defn}

The axiom of choice states that every epimorphism in the category $\cat{Set}$ of
sets and functions is a split epimorphism.

\begin{ex}
    Prove that a morphism $f$ that is both a monomorphism and a split
    epimorphism is necessarily an isomorphism.
    Argue by duality that a split monomorphism that is an epimorphism is also
    an isomorphism.
\end{ex}
\begin{proof}
    Let $g$ be the section of $f$.
    $f \circ g = \id$ because $r$ is a split epi; therefore,
    $f \circ g \circ f = f \circ \id$.
    But we can apply the left cancellation law because $f$ is a mono: $g \circ f = \id$.
    $f$ and $g$ witness the existence of an isomorphism.
    The second argument follows by duality.
\end{proof}

\begin{ex}
    Show that a morphism $f : X \rightarrow Y$ is a split epimorphism in a category
    $\mathbf{C}$ if and only if for all $W \in \cat{C}$, the post-composition function
    $f_\star~:~\Hom{C}{W}{X} \rightarrow \Hom{C}{W}{Y}$ is surjective.
\end{ex}
\begin{proof}
    Assume $f$ is a split epi and let $g : Y \rightarrowtail X$
    be the section of the splitting.
    We can assign to each $h : C \rightarrow Y$ the morphism
    $g \circ h : C \rightarrow X$ which is a preimage of $h$ under $f_\star$ because
    $f \circ g \circ h = \id \circ h = h$.

    Now, assume that $f_\star$ is surjective.
    Let $g : Y \rightarrow X$ be the preimage of $\id_Y$ under $f_\star$:
    therefore $f$ and $g$ witness a splitting that expresses $X$ as a retract of $Y$,
    with $f$ and $g$ being the retraction and the section respectively.
\end{proof}

\begin{defn}
    \label{defn:slice}
    For a category $\cat{C}$ and object $C \in \cat{C}$, there is a category $C/\cat{C}$
    called the \textbf{slice category} of $\cat{C}$ under $C$ whose objects are
    morphisms $f : C \rightarrow X$ and morphisms between morphisms
    $f : C \rightarrow X$ and $g : C \rightarrow Y$ are maps $h : X \rightarrow Y$
    such that $h \circ f = g$.
\end{defn}
The slice category of $\cat{C}$ over $C$ ($\cat{C}/C$) is the dual of $C/\cat{C}$.
The sometimes the slice category under $C$ is also called coslice category.
\begin{ex}
    Show that the category axioms hold for the slice categories $C/\cat{C}$ and $\cat{C}/C$.
\end{ex}

\section{Functors and natural transformations}
\begin{defn}
\label{defn:functor}
A \textbf{functor} $F : \cat{C} \longrightarrow \cat{D}$ between categories $\cat{C}$
and $\cat{D}$ is defined by:
\begin{itemize}
    \item an object $F X \in \cat{D}$ for each object $X \in \cat{C}$;
    \item a morphism $F f : F X \rightarrow F Y$ for each morphism $f: X \rightarrow Y$.
\end{itemize}
\end{defn}
The mapping should respect function composition and identity morphisms.
A functor $F : \cat{C}^{\op} \longrightarrow \cat{D}$ is also called a contravariant functor
between $\cat{C}$ and $\cat{D}$, while functors $F : \cat{C} \longrightarrow \cat{D}$ are
covariant. Functors may not preserve monomorphisms or epimorphisms, but they
necessarily preserve split monomorphisms, split epimorphisms, and isomorphisms.
Functor can be composed in the obvious way:
small categories (i.e. whose collections of objects and morphisms are sets) and functors
between them define the category $\cat{Cat}$.

\begin{defn}
\label{defn:hom}
Every locally small category $\cat{C}$ (i.e. whose hom-sets are actually sets and not proper
classes) defines for each object $X \in \cat{C}$ a pair of covariant and contravariant
functors $\Hom{C}{X}{-} : \cat{C} \longrightarrow \cat{Set}$ and
$\Hom{C}{-}{X} : \cat{C}^{\op} \longrightarrow \cat{Set}$.
The \textbf{hom-functors} map each object to a set of morphisms, and a morphism
$f : Y \rightarrow Z$ is subject to the following mappings:
\begin{itemize}
    \item $\Hom{C}{X}{f}$ is the post-composition function $f_\star$, which
        maps each morphism $g \in \Hom{C}{X}{Y}$ to the composite~$f \circ g \in \Hom{C}{X}{Z}$;
    \item $\Hom{C}{f}{X}$ is the pre-composition function $f^\star$, which
        maps each morphism $g \in \Hom{C}{Z}{X}$ to the composite
        $g \circ f \in \Hom{C}{Y}{X}$.
\end{itemize}
\end{defn}
These two functors together define a bifunctor $\Hom{C}{-}{-} : \cat{C}^\op \times \cat{C}
\longrightarrow \cat{Set}$ from the product category $\cat{C}^\op \times \cat{C}$ to
the category $\cat{Set}$ of sets and functions between them.

\begin{ex}
    Show that the hom-functors satisfy the functoriality axioms.
\end{ex}
\begin{proof}
        $\Hom{C}{X}{\id_Y} = \id_{\Hom{C}{X}{Y}}$ because the post-composition
            function $\id_\star$ is the identity by the category axioms; same
            for $\Hom{C}{\id_Y}{X}$.
        $\Hom{C}{X}{f \circ g} = (f \circ g)_\star = f_\star \circ g_\star =
            \Hom{C}{X}{f} \circ \Hom{C}{X}{g}$ by the associativity of composition;
            same for $\Hom{C}{f \circ g}{X}$.
\end{proof}

Every functor $F : \cat{C} \longrightarrow \cat{D}$ automatically defines another functor
$F : \cat{C}^\op \longrightarrow \cat{D}^\op$.

\begin{defn}
\label{defn:comma}
Given functors $F : \cat{D} \longrightarrow \cat{C}$ and $G : \cat{E} \longrightarrow \cat{C}$,
the \textbf{comma category} $F \downarrow G$ has
\begin{itemize}
    \item as objects, triples $(D \in \cat{D}, E \in \cat{E}, f : F D  \rightarrow G E)$;
    \item as morphisms $(D, E, f) \rightarrow (D', E', f')$, a pair
        $(h : D \rightarrow D', k : E \rightarrow E')$, such that the following diagram
        commutes.
        \begin{figure}[H]
            \centering
            \begin{tikzcd}
                F D \arrow[dd, "F h"'] \arrow[rr, "f"] &  & GE \arrow[dd, "G k"] \\
                                                       &  &                      \\
                F D' \arrow[rr, "f'"']                 &  & GE'                 
            \end{tikzcd}
        \end{figure}
\end{itemize}
Identity morphisms are $(\id_D, \id_E)$.
Composition of morphisms is defined in the obvious way, by stacking commutative
squares vertically.
\end{defn}
The functor $\dom : F \downarrow G \longrightarrow \cat{D}$ maps each object
$(D \in \cat{D}, E \in \cat{E}, f)$ to $D$ and
each morphism $(h, g)$ to $h$; the functor $\cod : F \downarrow G \longrightarrow \cat{E}$
maps each object $(D, E)$ to $E$ and each morphism $(h, g)$ to $g$.
The functoriality axioms are satisfied trivially.

\begin{ex}
    Define functors to construct the slice categories $C/\cat{C}$ and $\cat{C}/C$
    (Definition \ref{defn:slice}) as special cases of comma categories.
    What are the projection functors?
\end{ex}
\begin{proof}
    $C/\cat{C}$ can be expressed as the comma category $\Id \downarrow \1_C$, where
    $\Id : \cat{C} \longrightarrow \cat{C}$ is the identity functor on $\cat{C}$ and
    $\1_C : \cat{1} \longrightarrow \cat{C}$ is the functor from the trivial category
    $\cat{1}$ to $\cat{C}$ that maps the only object of $\cat{1}$ to $C$.
    $\cat{C}/C$ by duality is the comma category $\1_C \downarrow \Id$.
\end{proof}

In the same way as a functor is a mapping between categories, a natural transformation
is a structure-preserving mapping between functors.
\begin{defn}
    \label{defn:natural}
    A \textbf{natural transformations} $\alpha : F \Longrightarrow G$
    between functors $F : \cat{C} \longrightarrow \cat{D}$
    and $G : \cat{C} \longrightarrow \cat{D}$ is a mapping $\alpha_X$, called
    the component of $\alpha$ at $X$, for each $X \in \cat{C}$ such that for each
    morphism $f : X \rightarrow X'$ in $\cat{C}$ the
    following diagram commutes.
    \begin{figure}[H]
        \centering
        \begin{tikzcd}
            F X \arrow[dd, "Ff"'] \arrow[rr, "\alpha_X"] &  & GX \arrow[dd, "Gf"] \\
                                                         &  &                     \\
            F X' \arrow[rr, "\alpha_{X'}"']              &  & GX'                
        \end{tikzcd}
    \end{figure}
\end{defn}
A natural transformation is usually specified implicitly using its components, by
saying that morphisms $\alpha_X$ are natural. If a variable $X$ is being used
to index the $\alpha_X$, one may say that the arrows $\alpha$ are natural in $X$,
with $X$ ranging in the domain category of the functors.

Natural transformations can be composed in the obvious way, by composing the
respective components.

A natural transformations whose components are isomorphisms is called a
\textbf{natural isomorphism} $\alpha : F \cong G$, and gives another natural
transformation $\alpha^{-1}$ whose components are inverses to the components of $\alpha$.

The functions $f_\star$ and $f^\star$ for a
morphism $f : X \rightarrow Y$ in $\cat{C}$ are the components of two natural transformations
\begin{align*}
    f_\star : \Hom{C}{-}{X} \Longrightarrow \Hom{C}{-}{Y} \\
    f^\star : \Hom{C}{Y}{-} \Longrightarrow \Hom{C}{X}{-}
\end{align*}

\begin{ex}
    Recall the definition of the comma category of two functors
    $F : \cat{D} \longrightarrow \cat{C}$ and $G : \cat{E} \longrightarrow \cat{C}$
    (Definition \ref{defn:comma}). Construct a canonical natural transformation
    $\alpha : F \circ \dom \Longrightarrow G \circ \cod$.
\end{ex}
\begin{proof}
    The component of the natural transformation at the object
    $(D, E, f) \in F \downarrow G$ is given by $f$ itself.
    Then, $(F \circ \dom)(D, E, f) = FD$, $(G \circ \cod)(D, E, f) = GE$,
    $(F \circ \dom)(h, k) = Fh$ and $(G \circ \cod)(h, k) = k$.
    The naturality condition for $\alpha$ is exactly the same as the commuting square
    for the morphisms in the comma category.
\end{proof}

\begin{defn}
    \label{defn:cateq}
    An \textbf{equivalence of categories} is defined by two functors
    $F : \cat{C} \longrightarrow \cat{D}$ and $G : \cat{D} \longrightarrow \cat{C}$
    together with two natural isomorphisms $\eta : G \circ F \cong \Id_\cat{C}$ and
    $\epsilon : F \circ G \cong \Id_\cat{D}$.
\end{defn}

\begin{defn}
    \label{defn:funcprops}
    A functor $F : \cat{C} \longrightarrow \cat{D}$ is called:
    \begin{itemize}
        \item \textbf{faithful} if the function
            $$F(-) : \Hom{C}{X}{Y} \rightarrow \Hom{D}{FX}{FY}$$ is injective;
        \item \textbf{full} if $F(-)$ is surjective;
        \item \textbf{essentially surjective on objects} if for every object
            $Y \in \cat{D}$ there exists an object $X \in \cat{C}$ such that
            $FX$ is isomorphic to $Y$.
    \end{itemize}
\end{defn}
Note that fullness and faithfulness are local properties, i.e.,
a faithful functor need not be injective on objects or morphisms,
and a full functor need not be surjective on objects or morphisms.
A faithful functor that is injective on objects (and therefore on morphisms)
defines an \textbf{embedding} of $\cat{C}$ into $\cat{D}$; a fully faithful functor
that is injective on objects defines a \textbf{full embedding} of $\cat{C}$ as a
full subcategory of $\cat{D}$.

\begin{thm}
    \label{thm:cateq}
    A functor that determines an equivalence of categories is full, faithful and
    essentially surjective on objects.
    Assuming the axiom of choice, conversely, any functor with these properties
    defines an equivalence of categories.
\end{thm}
\begin{proof}
    Assume that $F : \cat{C} \longrightarrow \cat{D}$ and $G : \cat{D} \longrightarrow \cat{C}$
    determine an equivalence of categories via natural isomorphisms $\eta : G \circ F \cong \Id_\cat{C}$
    and $\epsilon : F \circ G \cong \Id_\cat{D}$.
    The fact that $F$ is essentially surjective is easily proved by observing that $D \simeq FGD$
    for all $D \in \cat{D}$ with the component $\epsilon_D$ as the witness of the isomorphism.

    Now note that for any morphism $f, g : X \rightarrow Y$ and fixed isomorphisms $X \simeq X'$,
    $Y \simeq Y'$, there is a unique morphism $f'$ such that the following diagram commutes:
    \begin{figure}[H]
        \centering
        \begin{tikzcd}
            X \arrow[d, "f"'] \arrow[r, "\simeq"] & X' \arrow[d, "\exists!f'"] \\
            Y \arrow[r, "\simeq"']                & Y'
        \end{tikzcd}
    \end{figure}
    Suppose that $Ff = Fg$, with $f, g : X \rightarrow Y$: then $GFf = GFg$. The
    components $\eta_X$, $\eta_Y$ act as the isomorphisms to construct the
    diagram
    \begin{figure}[H]
        \centering
        \begin{tikzcd}
            GFX \arrow[d, "GFf=GFg"'] \arrow[r, "\eta_X"] & X \arrow[d, "f\ \text{or}\ g"] \\
            GFY \arrow[r, "\eta_Y"']                     & Y
        \end{tikzcd}
    \end{figure}
    that commutes by the naturality of $\eta$.
    Both $f$ and $g$ make the diagram commute, and the uniqueness property guarantees
    that $f = g$.
    This proves the faithfulness of $F$ and, by symmetry, of $G$.

    Now assume $k : FX \rightarrow FY$. The following diagram shows the existence
    of a unique $h : X \rightarrow Y$ such that both $Gk$ and $GFh$ make the square commute:
    \begin{figure}[H]
        \centering
        \begin{tikzcd}
            X \arrow[d, "h"'] \arrow[r, "\eta_X"] & GFX \arrow[d, "GFh\ \text{or}\ Gk"] \\
            Y \arrow[r, "\eta_{Y}"']               & GFY
        \end{tikzcd}
    \end{figure}
    Therefore $GFh = Gk$, whence, by faithfulness of $G$, $Fh = k$.
    Thus $F$ is full, faithful and essentially surjective.

    For the converse, suppose that $F$ is a fully faithful essentially surjective functor.
    We shall first construct a functor $G$ in the following way:
    \begin{itemize}
        \item $GD$, with $D \in \cat{D}$, is chosen from the $C \in \cat{C}$ such that
            $FC \simeq D$.
            The existence of at least one such $C$ is guaranteed because $F$ is essentially
            surjective.
        \item $Gf$, with $f : X \rightarrow Y$ in $\cat{D}$, is constructed by composing $f$ with
            isomorphisms $\epsilon_X : X \simeq FGX$ and $\epsilon_Y : Y \simeq FGY$ given by the
            construction of $G$ and then taking the preimage of the resulting morphism
            $FGX \rightarrow FGY$ under $F$, whose existence is guaranteed by the fullness of $F$.
    \end{itemize}
    We need to prove the functoriality of $G$.
    Its action on identity morphisms is clear: to each object $X \in \cat{D}$ is assigned a single
    isomorphism $\epsilon_X : X \simeq FGX$; the application of $G$ to the identity morphism $\id_X$
    yields the preimage under $F$ of
    $\epsilon_X \circ \id_X \circ \epsilon_X^{-1} = \epsilon_X \circ \epsilon_X^{-1} = \id_{FGX}$.
    This preimage must be unique by the faithfulness of $F$ and therefore must be exactly $\id_{GX}$.

    $G(f \circ g)$ with $f : Y \rightarrow Z$ and
    $g : X \rightarrow Y$ is $F^{-1}(\epsilon_Z \circ f \circ g \circ \epsilon_X^{-1})$ (we use
    the notation $F^{-1}$ to indicate the inverse of the bijection estabilished by $F$ on each
    hom-set).
    But then $F(Gf \circ Gg)$ is
    \begin{align*}
        & F(F^{-1}(\epsilon_Z \circ f \circ \epsilon_Y^{-1}) \circ
        F^{-1}(\epsilon_Y \circ g \circ \epsilon_X^{-1})) & = \\
        & F(F^{-1}(\epsilon_Z \circ f \circ \epsilon_Y^{-1})) \circ
        F(F^{-1}(\epsilon_Y \circ g \circ \epsilon_X^{-1})) & = \\
        & (\epsilon_Z \circ f \circ \epsilon_Y^{-1}) \circ (\epsilon_Y \circ g \circ \epsilon_X^{-1}) & = \\
        & \epsilon_Z \circ f \circ g \circ \epsilon_X^{-1} &
    \end{align*}
    Thus $G(f \circ g)$ and $Gf \circ Gg$ are both valid preimages of
    $\epsilon_Z \circ f \circ g \circ \epsilon_X^{-1}$; by the faithfulness of $F$, these are equal
    and the functoriality of $G$ is proven.

    The isomorphisms $\epsilon_X$ are by definition the components of a natural isomorphism
    $F \circ G \cong \Id_\cat{D}$.
    It remains to exhibit a natural isomorphism $\eta : G \circ F \cong \Id_\cat{C}$.
    $F$ is full and faithful, so we may define the components $\eta_X$ as $F^{-1}\epsilon^{-1}_{FX}$.
    The preimage is guaranteed to exist because $\epsilon_{FX} : FX \simeq FGFX$.
    Fully faithful functors ``reflect'' isomorphisms, meaning that if $Ff$ is an isomorphism
    in $\cat{D}$, $f$ is necessarily an isomorphism in $\cat{C}$; this guarantees that components
    $\eta_X$ are isomorphisms. It remains to prove the naturality of $\eta$.

    Note that in a diagram constructed as follows, if the left square commutes and the entire rectangle
    commutes, it is easily shown that the right square commutes too (because $\gamma_1$ is an
    isomorphism and thus is also an epimorphism).
    \begin{figure}[H]
    \centering
    \begin{tikzcd}
        A \arrow[d, "f"'] \arrow[r, "\stackrel{\gamma_1}{\simeq}"] &
        B \arrow[d, "g"] \arrow[r, "\stackrel{\gamma_2}{\simeq}"] & C \arrow[d, "h"] \\
        A' \arrow[r, "\stackrel{\gamma'_1}{\simeq}"']              &
        B' \arrow[r, "\stackrel{\gamma'_2}{\simeq}"']             & C'
    \end{tikzcd}
    \end{figure}
    Now consider the following diagram in $\cat{D}$:
    \begin{figure}[H]
        \centering
        \begin{tikzcd}
            FX \arrow[d, "Ff"] \arrow[r, "\epsilon_{FX}"] &
            FGFX \arrow[r, "F\eta_X"] \arrow[d, "FGFf"] & FX \arrow[d, "Ff"] \\
            FY \arrow[r, "\epsilon_{FY}"']                &
            FGFY \arrow[r, "F\eta_Y"']                  & FY
        \end{tikzcd}
    \end{figure}
    The square on the left commutes by the naturality of $\epsilon_{FX}$ and the rectangle
    commutes by the definition of $\eta_X$. Therefore the right square commutes as
    well; the faithfulness of $F$ proves that $\eta_Y \circ GFf = f \circ \eta_X$,
    i.e., $\eta$ is natural.
\end{proof}
\end{document}

